%This should list each system interface and identify the functionality of the software to accomplish the system requirement and the interface description to match the system.
\subsection{User interfaces}
The users have several ways to access the system: a web application can be executed on any personal computer while a mobile application provides flexibility, portability and can be used literally everywhere. Despite the fact that the hardware interfaces running the application are rather different, a unified and common user interface is provided. It should be user friendly and very intuitive to allow everyone to easily use it without any specific knowledge.

Moreover the users have to interact with the on-board computer installed on each electric vehicle, therefore it should offer an interface as straightforward as the one implemented by the web and mobile applications.

\subsection{Hardware interfaces} \label{hw_int}
The web application can be executed on any general purpose computer that complies with the minimum system requirements specified in subsection \ref{ssec:hlimit}.

The mobile application has to exchange data with the GPS module located on any recent smart-phone. Moreover it has to access the mobile broadband in order to communicate with the main system server.

An on-board computer is already set up in each electric car and it talks to the vehicle control unit through the CAN bus and to the system server via the mobile broadband. All sensors and hardware components needed to support the functionalities of the system are already installed on the vehicles.

\subsection{Software interfaces}
The web based application must support the main browsers such as IE, Google Chrome and Mozilla Firefox.

The mobile application has to be compatible with iOS and Android. The server side of the application, that is the system back-end, stores data in a relational DBMS and runs on any web server supporting Java.

The on-board computers run a customized Linux-based kernel and provide adequate APIs for the developers to communicate easily with the existing vehicle systems. This is to make communication between the system-to-be and the existing software easier.

The back-end has mainly to deal with the rental service and data management, essentially aimed towards keeping track of transactions and customers information, as well as the cars states; this last information is tracked in order to notify an existing maintenance system about out-of-service cars.