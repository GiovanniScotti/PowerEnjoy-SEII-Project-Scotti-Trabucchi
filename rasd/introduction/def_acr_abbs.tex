\begin{description}
\item[API:] Application Programming Interface. A set of tools, protocols and libraries for building software and applications.
\item[Back-end application:] any computer program that remains in the background and offers application logic and communication interfaces to work with the front-end counterpart. It does not involve any graphical user interface, but it can provide a data access layer.
\item[Charging spot:] a particular parking spot equipped with a power grid and plugs to recharge the cars.
\item[CAN bus:] the Controller Area Network is a vehicle bus standard designed to allow microcontrollers and other devices to communicate with each other without a host computer.
\item[DBMS:] Database Management System.
\item[Driver:] See \textbf{User}.
\item[Front-end application:] any application the users interact with directly. It provides the so called presentation layer.
\item[GPS:] Global Positioning System.
\item[m:] meter, \emph{Interational System of Units (SI)} unit of measure.
\item[Mobile broadband:] it is the marketing term for wireless internet access delivered through mobile phone towers to any digital device using a portable modem.
\item[OS:] Operating System.
\item[Parking spot:] a generic place whitin the Safe Area where the driver can park the vehicles.
\item[PC:] Personal Computer.
\item[PIN:] Personal Identification Number.
\item[RASD:] Requirements Analysis and Specification Document.
\item[Ride:] the trip that involves the use of a \emph{PowerEnJoy} electric car.
\item[Safe Area:] the predefined area where it is possible to start and end a ride; it marks the boundaries within which the user can park a vehicle and conclude the ride so that the system stops charging him/her.
\item[System:] The software system-to-be, in all of its entirety.
\item[UMTS:] Universal Mobile Telecommunications System.
\item[User:] Any person subscribed to the service who rents a car using \hbox{\emph{PowerEnJoy}}.
\item[Vehicle:] Any of the electric cars provided by \emph{PowerEnJoy}.
\item[W3C:] World Wide Web Consortium. It is the main international standards organization for the World Wide Web, founded and led by Tim Berners-Lee.
\end{description}

