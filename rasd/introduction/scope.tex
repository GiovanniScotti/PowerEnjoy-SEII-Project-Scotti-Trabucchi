The product is a digital management system to support a car-sharing service that exclusively employs electric cars.

The system consists of a back-end server application that manages rental requests remotely and three front-end applications:

\begin{itemize}
\item A web-based application to provide the final user with a friendly interface to take advantage of the services of \hbox{\emph{PowerEnJoy}};
\item An application that runs on the existing on-board computers provided on each vehicle, used to interact with the car itself, unlock it and access the GPS/sat-nav service;
\item A mobile application that allows the user to easily access the service anywhere he/she needs to.
\end{itemize}
%serve per modellare i requisiti relativi al sw del computer di bordo [vd specifiche p.3 sez.5]
%"The user is notified of the current charges through a screen on the car"
The system is mainly intended for one type of user: drivers, who should be allowed to register and access the system via username and password, in order to make the renting and payment processes easier and quicker to carry out. Moreover, the system aids the users by locating nearby available vehicles and keeps track of the distance driven, all while notifying them about the amount of money they are being charged. A defined Safe Area stands as a limit for the service; properly equipped charging spots are signaled by the on-board computers.

The system aims to motivate drivers to maintain a virtuous behavior providing discounts when it detects signs of responsible and ecologic actions.

Lastly, the system is also in charge to inform - by marking the vehicles as out-of-service when necessary - the existing maintenance system about out-of-order vehicles so that the technicians can work out any kind of issue; the existing maintenance system will be able to do so by simply accessing part of the database of the system-to-be.