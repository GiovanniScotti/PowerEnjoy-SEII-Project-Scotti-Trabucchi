The following sections aim to describe the integration testing sequence of the different components and subsystems of \emph{PowerEnJoy}. From now on the following notation will be used: C1 $\rightarrow$ C2 indicates that C2 is necessary for C1 to work properly.

\subsection{Software Integration Sequence}\label{soft_int_seq}
The components of each subsystem are tested starting from the most to the least independent one.

\subsubsection{Data Access}
%DBMS <- JEBs
The first components to be integrated are those relative to the data access, starting from the database core: the DBMS. This will be integrated with all the Java Entity Beans (JEB) defined in the Design Document~\cite{dd}. 

In order to do so, the DBMS will need a driver for each Entity Bean to carry out queries and verify their correctness on a dummy database, containing a greatly reduced number of test information. Said test database will be structured based on the E-R schema that will be adopted for the final implementation of the data layer.

\begin{figure}[H]
\begin{center}
		\includegraphics[width=\textwidth]{./integration_strategy/diagrams/data_access.png}
		\caption{The integration of the system components at a Data Access level. The detailed integration steps are described at the end of the section in an overall comprehensive diagram.}
\end{center}
\end{figure}

%JEB User <- UserManager
%JEB User <- SecurityAuthenticator
%JEB Car <- CarStatusManager
%JEB Car <- ReservationManager
%JEB Car <- MapManager
%JEB Reservation <- ReservationManager
%JEB Reservation <- SecurityAuthenticator
%JEB Ride <- RideManager
%JEB SafeArea <- MapManager
%JEB PowerGridStation <- MapManager
%JEB AlternativeChargesSituation <- DiscountProvider
%JEB Payment <- PaymentGateway
\noindent
The next steps involve the integration of the Session Beans which take advantage of said Entity Beans and are in charge of accessing them in the final application.

\subsubsection{User and Utilities Management}
%NotificationManager <- UserManager
%NotificationManager <- PaymentGateway
%UserManager <- PaymentGateway
The integration can begin by covering the user management and the business logic utilities, that are considered relevant to support the rest of the application functionalities. To begin with, the most independent bean is, in this case, the NotificationManager, that requires two drivers to invoke and test methods later used by UserManager and PaymentGateway (1).
\noindent
The UserManager component can then in turn be integrated (2), using a driver to be replaced with the PaymentGateway together with the previously mentioned one, and to call the needed methods appropriately for the case.

\begin{figure}[H]
\begin{center}
		\includegraphics[width=0.8\textwidth]{./integration_strategy/diagrams/user_utilities.png}
\end{center}
\end{figure}

\subsubsection{Payment Management}
%DiscountProvider <- RideManager
%DiscountProvider <- PaymentGateway
%PaymentGateway <- RideManager
%PaymentGateway <- ReservationManager
The payment management context can be covered next in the integration process, starting once more from the most independent component, represented by the DiscountProvider. This will need two different drivers, one for the methods called by the RideManager and another for the PaymentGateway (1).
\noindent
The PaymentGateway itself can in turn be integrated by using two other drivers, that will be replaced respectively with the RideManager and ReservationManager (2). The two drivers will call methods related with ride payments (RideManager) and fee payments (ReservationManager).

\begin{figure}[H]
\begin{center}
		\includegraphics[width=0.8\textwidth]{./integration_strategy/diagrams/payment.png}
\end{center}
\end{figure}

\subsubsection{Ride and Reservation Management}
%CarStatusManager <- RideManager	
%CarStatusManager <- ReservationManager
%RideManager <- SecurityAuthenticator
%SecurityAuthenticator <- MapManager
The most critical features of the application revolve around the management of rides and reservations. Within this context, the most independent functionality is provided by the CarStatusManager bean, since no other bean depends on it apart from the already integrated Entity Beans. In order to integrate the CarStatusManager, there will be the need of two drivers: invoking methods in place of the ReservationManager and of the RideManager (1).
\noindent
In its turn, the RideManager itself needs a driver in order to be integrated (2), which will represent the SecurityAuthenticator session bean and call the methods exposed by RideManager in its place.
\noindent
The last component to be integrated in this context is the SecurityAuthenticator, which will need a driver to simulate method calls by the MapManager bean on it (3).

\begin{figure}[H]
\begin{center}
		\includegraphics[width=0.8\textwidth]{./integration_strategy/diagrams/ride_reservation.png}
\end{center}
\end{figure}

\subsubsection{Application Logic Overall Integration}
%UserManager <- UserManagementContainer
%MapManager <- UtilitiesContainer
%NotificationManager <- UtilitiesContainer
%DiscountProvider <- ChargesManagementContainer
%PaymentGateway <- ChargesManagementContainer
%ReservationManager <- RideManagementContainer
%RideManager <- RideManagementContainer
%SecurityAuthenticator <- RideManagementContainer
%CarStatusManager <- RideManagementContainer
To conclude the integration process for the application logic tier, drivers for the EJB Containers must be provided, in order to have a means to emulate multiple requests for session bean instances; this will help in testing the underlying system effectiveness in managing heavy loads and concurrency during ordinary activity.
\noindent
Hence, in order to simulate the correctness of requests to the individual containers, a driver for each individual container must be used to bypass the runtime behaviour and reproduce said requests in a deterministic way. This approach will avoid the necessity of implementing the whole system before having the possibility to test the correctness of the requests to the containers.

\begin{figure}[H]
\begin{center}
		\includegraphics[width=\textwidth]{./integration_strategy/diagrams/containers.png}
		\caption{The integration of the system components at a container level. The detailed integration steps are described at the end of the section in an overall comprehensive diagram.}
\end{center}
\end{figure}

\subsubsection{On-Board Application}
%GPSManager <- ApplicationController
%UIManager <- ApplicationController
%DataProcessingUnit <- ApplicationController
%ConnectivityUnit <- ApplicationController
With respect to the On-Board Application, the integration process must proceed for each of the base components individually with the application controller.

\begin{figure}[H]
\begin{center}
		\includegraphics[width=\textwidth]{./integration_strategy/diagrams/on_board.png}
\end{center}
\end{figure}

\subsubsection{Mobile Application}
%GPSManager <- ApplicationController
%UIManager <- ApplicationController
Similarly to what is stated for the on-board application, the mobile application will follow the order imposed by the centrality of the application controller; the other components will be integrated individually with the controller itself.

\begin{figure}[H]
\begin{center}
		\includegraphics[width=0.6\textwidth]{./integration_strategy/diagrams/mobile.png}
\end{center}
\end{figure}

\subsubsection{Web Application}
%JavaServerPages <- WebController
As far as the Web Application is concerned, the Java Server Pages will be integrated with the web controller, which will be in turn integrated with the container in which it is defined.

\begin{figure}[H]
\begin{center}
		\includegraphics[width=0.8\textwidth]{./integration_strategy/diagrams/web.png}
\end{center}
\end{figure}

\noindent
An overall view of the integration sequence is provided below: \textcolor{blue}{blue} arrows highlight interactions between EJBs and the DBMS, \textcolor{red}{red} arrows highlight interactions between Containers and the respectively contained Session Beans.

\begin{figure}[H]
\begin{center}
		\includegraphics[width=\textwidth]{./integration_strategy/diagrams/overall.png}
		\caption{Overall diagram of the components integration. Numbers point out the integration order.}
\end{center}
\end{figure}

%{p{0.03\textwidth} | p{0.29\textwidth} | p{0.40\textwidth} | p{0.40\textwidth}}

\newpage
\begingroup
\setlength{\LTleft}{-20cm plus -1fill}
\setlength{\LTright}{\LTleft}
\begin{longtable}{p{0.03\textwidth} | p{0.29\textwidth} | p{0.40\textwidth} | p{0.40\textwidth}}
\hline
\textbf{N.} & \textbf{Subsystems} & \textbf{Component} & \textbf{Integrates with} \\
\hline
I01 & Database, App. Logic & (JEB) User & DBMS \\
\hline
I02 & Database, App. Logic & (JEB) Car & DBMS \\
\hline
I03 & Database, App. Logic & (JEB) Reservation & DBMS \\
\hline
I04 & Database, App. Logic & (JEB) Ride & DBMS \\
\hline
I05 & Database, App. Logic & (JEB) SafeArea & DBMS \\
\hline
I06 & Database, App. Logic & (JEB) PowerGridStat. & DBMS \\
\hline
I07 & Database, App. Logic & (JEB) AlternativeChargesSit. & DBMS \\
\hline
I08 & Database, App. Logic & (JEB) Payment & DBMS \\
\hline
I09 & App. Logic & (SB) DiscountProvider & (JEB) Alternative ChargesSit. \\
\hline
I10 & App. Logic & (SB) CarStatusManager & (JEB) Car \\
\hline
I11 & App. Logic & (SB) UserManager & (JEB) User \\
									& & & (SB) NotificationManager \\
\hline
I12 & App. Logic & (SB) PaymentGateway & (JEB) Payment \\
									   & & & (SB) UserManager \\ 
									   & & & (SB) DiscountProvider \\ 
									   & & & (SB) NotificationManager \\
\hline
I13 & App. Logic & (SB) RideManager & (JEB) Ride \\
									& & & (SB) PaymentGateway \\
									& & & (SB) DiscountProvider \\
									& & & (SB) CarStatusManager \\
\hline
I14 & App. Logic & (SB) ReservationManager & (JEB) Car \\
										   & & & (JEB) Reservation \\
										   & & & (SB) CarStatusManager \\
										   & & & (SB) PaymentGateway \\
\hline
I15 & App. Logic & (SB) SecurityAuthenticator & (JEB) User \\
											  & & & (JEB) Reservation \\
											  & & & (SB) RideManager \\
\hline
I16 & App. Logic & (SB) MapManager & (JEB) Car \\
								   & & & (JEB) SafeArea \\ 
								   & & & (JEB) PowerGridStation \\ 
								   & & & (SB) SecurityAuthenticator \\
\hline
I17 & App. Logic & (Cont.) UserManagementCont. & (SB) UserManager \\
\hline
I18 & App. Logic & (Cont.) UtilitiesContainer & (SB) MapManager \\
											  & & & (SB) NotificationManager \\
\hline
I19 & App. Logic & (Cont.) ChargesManagementCont. & (SB) PaymentGateway \\
													  & & & (SB) DiscountProvider \\
\hline
I20 & App. Logic & (Cont.) RideManagementCont. & (SB) RideManager \\ 
												   & & & (SB) ReservationManager \\
												   & & & (SB) SecurityAuthenticator \\
												   & & & (SB) CarStatusManager \\
\hline
I21 & On-Board Client & ApplicationController & UIManager \\
											  & & & GPSManager \\
											  & & & ConnectivityUnit \\
											  & & & DataProcessingUnit \\
\hline
I22 & Mobile Client & ApplicationController & UIManager \\
											& & & GPSManager \\
\hline
I23 & Web & WebController & JavaServerPages \\
\hline
I24 & Web & WebContainer & WebController \\
\hline
\caption{Integration order of the system components.}
\label{software_int}
\end{longtable}
\endgroup

\subsection{Subsystem Integration Sequence}
The integration sequence of the high-level subsystems is described in Figure \ref{h_level_subsys} and Table \ref{subsys_int}.

\begin{figure}[H]
\begin{center}
		\includegraphics[width=\textwidth]{./integration_strategy/diagrams/h_level_subsys.png}
		\caption{Diagram representing the order of the subsystems integration.}
		\label{h_level_subsys}
\end{center}
\end{figure}

\begin{table}[H]
\begin{center}
\begin{tabular}{p{0.05\textwidth} | p{0.3\textwidth} | p{0.38\textwidth}}
\hline
\textbf{N.} & \textbf{Subsystem} & \textbf{Integrates with} \\
\hline
SI1 & Application Logic Tier & Database Tier \\
\hline
SI2 & Application Logic Tier & (EXT) MaintenanceSystem \\
\hline
SI3 & Application Logic Tier & (EXT) PaymentHandler \\
\hline
SI4 & On-Board Application & Application Logic Tier \\
\hline
SI5 & Mobile Application & Application Logic Tier \\
\hline
SI6 & Web Tier & Application Logic Tier \\
\hline
SI7 & Web Browser & Web Tier \\
\hline
\end{tabular}
\end{center}
\caption{Integration order of the subsystems described in Section \ref{elems_int}. External system interfaces to be integrated with \textit{PowerEnJoy}'s subsystems are marked with (EXT).}
\label{subsys_int}
\end{table}

Note that the base for the subsystem integration is the data tier, which is considered the most critical component; for the same reason, the application logic tier comes before all kinds of clients, since a working business logic is mandatory to have properly functioning clients. The choice of integrating the on-board application before other clients is due to the critical-module-first approach that has been chosen for this step of the integration process, since the on-board functionalities are meant to be core for the application itself. Lastly, the mobile application will be integrated first, since the integration of the web tier and browser client is heavier and more complex; moreover, this choice will allow the development team to have a working part of the system implementing a client-server structure even before having fully developed the web application.

\subsubsection{External Systems}
%STUB OF PAYMENT HANDLER ENDPOINT <- PaymentGateway
%CarStatusManager <- DRIVER FOR MAINTENANCE SYSTEM ENDPOINT
As stated in Section \ref{strategy} of this document, the relevance of the interactions with external systems makes it necessary to integrate some of said functionalities at an application logic level.
\noindent
To be precise, the components to be integrated are the endpoints of the Payment Handler and of the Maintenance System. Since in the final implementation of the application the Payment Handler will provide the APIs to interface with it, the integration will need a \emph{stub} of the Payment Handler endpoint, which will simulate the behaviour of the external payment system upon being invoked for any function. The Maintenance System will instead use the APIs provided by the \textit{PowerEnJoy} system itself, hence the integration process will need a \emph{driver} to simulate its calls to the \textit{PowerEnJoy} system.

\begin{figure}[H]
\begin{center}
		\includegraphics[width=0.8\textwidth]{./integration_strategy/diagrams/external_systems.png}
		\caption{Integration of the external systems.}
\end{center}
\end{figure}