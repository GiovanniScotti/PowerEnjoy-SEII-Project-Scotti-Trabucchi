As far as the integration testing process is concerned, a \textbf{bottom-up approach} will be followed. 

The choice of the bottom-up testing strategy is natural since the integration testing can start from the smallest and lowest-level components, that already have unit tests and do not depend on other components or not-already-developed components. In this way the total amount of needed stubs to accomplish the integration will be deeply reduced, but temporary programs for higher-level modules (drivers) are used to simulate them and invoke the unit under test.

The bottom-up strategy will be mixed with a \textbf{critical-module-first approach}, in order to avoid issues related to the failures of core components and to pose a threat to the correct implementation of the entire \emph{PowerEnJoy} system.

Moreover the higher-level subsystems described in section \ref{elems_int} are loosely coupled and fairly independent from one another because they correspond to different tiers. In this case, the critical-module-first approach is used to establish the integration order to obtain the full system.

Notice that the DBMS is a commercial component already developed that can be used directly in the bottom-up approach and does not have any dependency.
% CE NE SONO ALTRI?