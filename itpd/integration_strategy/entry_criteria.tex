This section expresses the prerequisites needed to be met before the integration phase takes place.

\begin{description}
\item[Documentation:] The documentation for every method and class must be provided for each individual component, in order to make it easier to reuse classes and understand their functioning; this is in fact also a prerequisite for the unit tests to be performed before the integration test phase. When necessary, a formal language specification of the classes' behaviours can be used (such as JML - Java Modelling Language).
\item[Unit tests:] All the classes and methods must be tested thoroughly using JUnit, in order to assure a properly correct behaviour of the internal mechanics of the individual components. It is required that the test coverage of each class and package reaches 90\% of the code lines; moreover, test cases must be written with continuity and executed at every consecutive build of the project: this is needed in order to ensure that newly added lines do not interfere with the stability of the rest of the code.
\item[Code Inspection and Analysis:] Both automated data-flow analysis and code inspection must be performed on the whole project classes. This will reduce the risk that, during the integration test phase, any code-related issues or bugs rise, leading to more complex problematic situations to be solved in latter phases of the project development, with much greater effort for the development team.
\item[RASD and DD:] Along with the indications provided in this very document (ITPD), the two previous documents for this project, RASD and DD, must be delivered before the integration test phase can begin.
\end{description}