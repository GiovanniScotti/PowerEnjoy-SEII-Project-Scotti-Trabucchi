As stated in the RASD~\cite{rasd}, the system must be aware of different discount and additional charges situations that can occur during one user's ride and bring respectively a reduction or an increase of the final charges at the end of the ride.

The \textbf{discount situations} that the system must recognize are the following:
\begin{enumerate}
\item if the user takes at least two other passengers on the car, the system applies a discount of 10\% on the final charges.
\item if a ride ends with no more than 50\% of the car battery empty, the system applies a discount of 20\%.
\item if the user plugs the car into the power grid at the end of the ride, the system applies a discount of 30\%.
\end{enumerate}
\noindent
The \textbf{additional charges situations} that are detected by the system are the following:
\begin{enumerate}
\item if a car is left more than 3km away from the nearest power grid, the system charges 30\% more on the last ride.
\item if a car is left with no more than 20\% of battery level and the user does not plug it into the power grid, the system charges 30\% more.
\end{enumerate}
\noindent
The on-board application is aware of all these situations and it is able to detect them thanks to the sensors the car is provided.
As soon as a situation occur, the on-board application gets in touch with the Application Server. The business logic is in charge of:
\begin{enumerate}
\item understanding if the situation will bring a discount or additional charges;
\item saving in a \emph{List} of additional charges situations the event that has just happened if it is an additional charges situation;
\item saving the event in an \emph{attribute} if it is a discount situation and the discount percentage is greater than the previous saved situation or the attribute is NULL.
\end{enumerate}
\noindent
After 5 minutes to the end of the ride, in order to be specifications-compliant, the system must calculate the final charges taking into account the previously saved situations:
\begin{enumerate}
\item \emph{foreach} element of the additional charges List, the increment to the final charges is computed as: $standard charges * situation percentage$, where \emph{standard charges} are the charges at the end of the ride without any discount/additional charges situation;
\item all the increments must be summed to the \emph{standard charges};
\item if at least one discount situation happened during the ride, this will lead to a decrement of the final charges that is equal to $standard charges * situation percentage$.
\end{enumerate}
\noindent
Note that \emph{situation percentage} is the discount/additional charges percentage divided by 100. 