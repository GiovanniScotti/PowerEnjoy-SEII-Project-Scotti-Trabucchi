\subsection{Authentication Method}
As soon as the user unlocks the car, he/she needs to authenticate him/herself again to start the ride. This is a remarkable step introduced for security purposes: the goal is to prevent someone who has somehow stolen the password of a registered user to take the advantage of the service provided by \emph{PowerEnJoy}.

The authentication method will be implemented as a PIN provided by the system during the registration process that cannot be changed by the user as stated in the RASD document~\cite{rasd}.

The above-mentioned PIN must be inserted via the on-board application and must be checked by the system before unlocking the engine of the car.

\subsection{User passwords storage}
For security reasons, the user's password is stored using cryptographic hash functions. In addition to that, the password is not only hashed, but also salted. This is a common security choice since many users reuse passwords for multiple sites and a cyber-attack could jeopardize their sensitive information.

\subsection{Maps}
In order to support the car finding functionality, the system will make use of an external maps service: \textit{Google Maps}. This choice is motivated by the fact that manually developing a map service is not a viable solution with respect to the size of the project.

\textit{Google Maps}'s services will be used to translate addresses to coordinates and vice-versa during the process of finding a car: the translation from address to position will be performed when the location input by a user is an address; the inverse translation will happen when the system needs to display the address corresponding to the position of a reserved car.

Another case in which the services of \textit{Google Maps} will be used is to graphically show the map during the same process.