%Inspection results
This chapter contains the result of the code inspection carried out on the assigned class and its methods. All the points of the given checklist~\cite{assignment} have been checked.
\section{Notation}
The following notations have been used to draw up this document:
\begin{itemize}
\item A specific line of code is referred as follows: \textbf{L.}123
\item Specific items of the code inspection checklist~\cite{assignment} are referred as follows: \textbf{C1}, \textbf{C2}, ... \textbf{Cn}
\item An interval of lines of code is referred as follows: \textbf{L.}123-\textbf{L.}456
\end{itemize}

\section{Issues}
\subsection{ModelDataFileReader Class}
\begin{enumerate}
\item \textbf{C18.} Comments are NOT used to adequately explain what the class, blocks of code and methods do except for the following methods:
	\begin{itemize}
	\item \texttt{getDataFilesNames()}
	\item \texttt{getDataFileNamesIterator()}
	\item \texttt{getModelDataFile(String dataFileName)}
	\end{itemize}
	Moreover the provided JavaDoc is very limited.
\item \textbf{C22.} JavaDoc for the public method \texttt{getModelDataFiles} is missing, therefore it is impossible to check whether the interface has been implemented consistently or not.
\item \textbf{C40.} No object is compared to another one. There are only objects compared to \texttt{null} by using the "==" operator correctly.
\item \textbf{C42.} The following lines contain log error messages that do not provide guidance or hints on how to correct the problem at all: \textbf{L.}102, \textbf{L.}115, \textbf{L.}129-130, \textbf{L.}136-137, \textbf{L.}142-143, \textbf{L.}158-159.
\end{enumerate}

%ATTENZIONE:
%The style used is the Kernigham and Ritchie one.
%Controllare C11: L.49-50 (è pieno anche più avanti)

%C13-C14: Da dove si parte a misurare la lunghezza della linea?
%C15: Good code conventions say that the line break occurs BEFORE an operator and AFTER a comma
%C16: cosa vuol dire higher-level breaks?

%L266: error in the article ("a Iterator")
%L276: error in the article ("an DataFile")

%PUNTI Controllati da Gio:
%2-4-6-8-10-12-18-20-(22)-24-26-40-42-48-50-52-54-58-60