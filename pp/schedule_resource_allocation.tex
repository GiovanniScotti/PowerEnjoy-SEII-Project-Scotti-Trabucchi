\section{Tasks and Schedule}
The main tasks of which the \textit{PowerEnJoy} project is composed of are the following:
\begin{enumerate}
\item \textbf{Requirements Analysis and Specification Document (RASD)} delivery - deliver a document containing the description of all goals, domain assumptions, functional and non-functional requirements for the project;
\item \textbf{Design Document (DD)} delivery - deliver a document describing the architectural design of the software system to be produced;
\item \textbf{Integration Test Plan Document (ITPD)} delivery - deliver a document containing the strategy to perform integration testing on the components of the system;
\item \textbf{Project Plan Document (PPD)} delivery - deliver a document containing the description of the schedule and tasks for the project and an estimation of the effort and size of the project itself, as well as an analysis of the risks that the project could face during its life-cycle;
\item \textbf{Implementation} - implement the software product and thoroughly write unit tests for all the code;
\item \textbf{Integration testing} - test the integration of all the software components of the project;
\end{enumerate}

During the project life-cycle, some of these task can not begin if others are not completed yet. To illustrate the precedence constraints of the case, a \textbf{dependency graph} is provided in Figure %\ref{•}.

Note that, however, since a software project is highly subject to change in requirements and evolves continuously, there might be the need of coming back to previous tasks one or more times after the conclusion of said tasks themselves.

%insert DAG dependency graph here

The following is a Gantt chart to describe the chosen schedule for the project:

%gantt chart here

\section{Resource Allocation}