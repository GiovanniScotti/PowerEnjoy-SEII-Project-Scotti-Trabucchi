\section{Purpose and Scope}
The purpose of this document is to provide a reasonable estimate of the complexity of the \textit{PowerEnJoy} project in terms of the development team effort.

In the first section of the document, two complementary estimation models, \textit{Function Point Analysis (FPA)} and \textit{COCOMO II}, are going to be used in order to support the estimation process; the result will be an estimate of the number of lines of source code (SLOC) to be written and the average effort required for the development process itself.

In the second section, the organizational structure of the project plan will be laid down; here will be defined a possible schedule to cover all activities within the deadline and with a proper redistribution of tasks among the team members.

The last section of the document contains an analysis of all kinds of risks that the project could be facing throughout its life-cycle, from the development phase to the deployment phase, up to the maintenance and support phase. Here are also defined the safety measures, both reactive and pre-emptive, to face the eventual occurrence of the risks mentioned above.

\section{Definitions, Acronyms and Abbreviations}
\begin{description}
\item[RASD:] Requirements Analysis and Specification Document
\item[DD:] Design Document
\item[ITPD:] Integration Test Plan Document
\item[FP:] Function Point
\item[FPA:] Function Point Analysis
\item[COCOMO:] COnstructive COst MOdel
\item[SLOC/KSLOC:] Source Lines Of Code / Kilo Source Lines Of Code
\item[ILF:] Internal Logic Files
\item[EIF:] External Interface Files
\item[EI:] External Inputs
\item[EO:] External Outputs
\item[EQ:] External Inquiries
\end{description}

\section{Reference Documents}
The indications provided in this document are based on the ones stated in the previous deliverables for the project, the RASD document~\cite{rasd}, the DD document~\cite{dd} and the ITPD document~\cite{itpd}.

Moreover it is strictly based on the specifications concerning the RASD assignment~\cite{se-assignments} for the Software Engineering II project, part of the course held by professors Luca Mottola and Elisabetta Di Nitto at the Politecnico di Milano, A.Y. 2016/17.

To support the application of the COCOMO II model estimate, the COCOMO II Model Definition Manual~\cite{cocomo-manual} has been followed.