\section{Scale Drivers}
Some of the most important factors contributing to the duration and cost of a project are the Scale Drivers. Each of said drivers describes the project itself and determines the exponent used in the Effort Equation. More precisely, the Scale Drivers reflect the non-linearity of the effort with relation to the number of SLOC. Each scale driver has a range of rating levels from \textit{Very Low} to \textit{Extra High}. The result of the evaluation is summed up in Table \ref{scale_drivers}.

\begin{description}
\item[Precedentedness:] points out the previous experience of the team with the organization and development of large scale projects. Since the team members are new to most of the notions concerning this kind of projects, the Precedentedness is \textit{LOW}.
\item[Development Flexibility:] reflects the level of flexibility in the development process with respect to the given specifications and requirements. Since the project is bounded to a set of prescribed specifications, but a certain degree of flexibility is allowed in the definition of the requirements, the Development Flexibility is \textit{NOMINAL}.
\item[Risk Resolution:] reflects the level of awareness and responsiveness with respect to risks. A rather detailed risk analysis, followed by possible countermeasures, is offered in Section \ref{risk_man}. Therefore, the Risk Resolution driver is set to \textit{HIGH}.
\item[Team Cohesion:] indicates how good the relationship among the team members is and how well the development team works together. Since the team is effective and cohesive with no communication problems, the parameter is set to \textit{VERY HIGH}.
\item[Process Maturity:] by means of a set of levels, it describes how well the behaviours, practices and processes of an organization can reliably produce required outcomes. It is set to \textit{CMM Level 3,} which corresponds to a defined level with respect to the software development process of our organization. At this level the organization has developed its own standard software process and has a greater attention to documentation, standardization and integration. In the COCOMO II model, this corresponds to a level of \textit{HIGH}.
\end{description}

\begin{table}[H]
    \centering
    \begin{tabular}{p{0.45\textwidth}|p{0.35\textwidth}|p{0.1\textwidth}}
        \hline
        \multicolumn{1}{c |}{\textbf{Scale Driver}} & \multicolumn{1}{c |}{\textbf{Factor}} & \multicolumn{1}{c}{\textbf{Value}} \\
        \hline
        \hline
        Precedentedness (PREC) & Low & 4.96 \\
        \hline
        Development Flexibility (FLEX) & Nominal & 3.04 \\
        \hline
        Risk Resolution (RESL) & High & 2.83 \\
        \hline
        Team Cohesion (TEAM) & Very High & 1.10 \\
        \hline
        Process Maturity (PMAT) & High & 3.12 \\
        \hline
        \textbf{Total} & $E=0.91 + 0.01 \times \sum_{i}SF_i$ & 1.0605 \\
        \hline
    \end{tabular}
    \caption{Result of the scale drivers analysis.}
    \label{scale_drivers}
\end{table}

\section{Cost Drivers}
Cost Drivers appear as parameters of the effort equation reflecting characteristics of the developing process and acting as multiplication factors on the effort needed to carry out said project.

Since this effort analysis is carried out at the beginning of the life-cycle of the project, before the writing of the RASD~\cite{rasd} and the DD~\cite{dd}, the point-of-view will be that of the \textit{early design} version of the COCOMO II model. Clear information about the architecture are still not available and the team is exploring and evaluating different architectural alternatives.

In order to conduct the analysis of said cost drivers, guidelines given by the COCOMO Model Definition Manual~\cite{cocomo-manual} have been followed.

Each cost driver of the post-architecture approach has a rating that is considered as a numerical value during the early design analysis (Very Low = 1, Extra High = 6). Ratings related to the Post-Architecture cost drivers corresponding to an Early Design one will be summed and converted into rating levels using the provided tables.

%PERS
%RCPX
%RUSE
%PDIF
%PREX
%FCIL
%SCED
\begin{description}
\item[Personnel Capability:] describes the overall capabilities of the team members in terms of problem-solving, actual implementation skills and ratings of personnel turnover. It derives from the conjunction of the Analyst Capability (ACAP) factor, the Programmer Capability (PCAP) factor and the Personnel Continuity (PCON) factor. In detail, the ACAP factor expresses the analysis and design ability of the team members, as well as the ability to communicate and cooperate. It is set to Nominal (3) since the team members have no previous experience in finding requirements, but communication and cooperation is very high. PCAP is set to Nominal (3) too, because the programming abilities of the team is in the average. PCON is instead set to Very High (5) since the turnover is totally absent.

The PERS factor is set to High according to the following table:

\begin{table}[H]
	\begin{adjustwidth}{-.5in}{-.5in}
    \centering
    \begin{tabular}{p{4cm}|p{1cm}|p{1cm}|p{1cm}|p{1.5cm}|p{1cm}|p{1cm}|p{1cm}}
        \hline
        Sum of ACAP, PCAP, PCON Ratings & 3,4 & 5,6 & 7,8 & 9 & 10,11 & 12,13 & 14,15 \\
        \hline
        \hline
        Rating Levels & Extra Low & Very Low & Low & Nominal & High & Very High & Extra High \\
        \hline
        Effort Multipliers & 2.12 & 1.62 & 1.26 & 1.00 & 0.83 & 0.63 & 0.50 \\
        \hline
    \end{tabular}
    \caption{PERS cost driver.}
    \end{adjustwidth}
\end{table}

\item[Product Reliability and Complexity:] this Early Design cost driver depends on the characteristics of the product under development and combines the following four Post-Architecture cost drivers: Required Software Reliability (RELY), Database Size (DATA), Product Complexity (CPLX), and Documentation Match to Life-cycle Needs (DOCU). RELY is set to Nominal (3) since a software failure brings moderate financial losses, but there is no risk to human life. DATA instead is set to Very High (5) since the size of the testing database can be estimated to be 100MB while the number of SLOC has been computed in the Section about FPs. All this leads to a $D/P$ parameter (Test Database Size over SLOC ratio) that is grater than 1000~\cite{cocomo-guide}. CPLX is set to Nominal (3) because no complex algorithm are involved and I/O processing includes status checking and error processing. Moreover, the user interface will be of a medium complexity as no complicated widgets are planned to be used. DOCU is set to Nominal (3) too, since the team does not want to spend excessive time on documentation, but a right-sized one is very important to reduce system maintenance costs and improve the software understanding.

The RCPX factor is set to High according to the following table:

\begin{table}[H]
	\begin{adjustwidth}{-.5in}{-.5in}
    \centering
    \begin{tabular}{p{4cm}|p{1cm}|p{1cm}|p{1cm}|p{1.5cm}|p{1cm}|p{1cm}|p{1cm}}
        \hline
        Sum of RELY, DATA, CPLX, DOCU Ratings & 5,6 & 7,8 & 9-11 & 12 & 12-15 & 16-18 & 19-21 \\
        \hline
        \hline
        Rating Levels & Extra Low & Very Low & Low & Nominal & High & Very High & Extra High \\
        \hline
        Effort Multipliers & 0.49 & 0.60 & 0.83 & 1.00 & 1.33 & 1.91 & 2.72 \\
        \hline
    \end{tabular}
    \caption{RCPX cost driver.}
    \end{adjustwidth}
\end{table}

\item[Re-usability:] this cost driver points out the additional effort needed to construct components intended to be reused on current or future projects. A Nominal value is set because re-usability is actually limited to the project itself and no similar projects are planned for the future. 

\begin{table}[H]
	\begin{adjustwidth}{-.5in}{-.5in}
    \centering
    \begin{tabular}{p{3.8cm}|p{1cm}|p{1cm}|p{1.5cm}|p{1.6cm}|p{1.5cm}|p{2cm}}
    	\hline
        RUSE Descriptors: & & none & across project & across program & across product line & across multiple product lines \\
        \hline
        Rating Levels & Very Low & Low & Nominal & High & Very High & Extra High \\
        \hline
        Effort Multipliers & n/a & 0.95 & 1.00 & 1.07 & 1.15 & 1.24 \\
        \hline
    \end{tabular}
    \caption{RUSE cost driver.}
    \end{adjustwidth}
\end{table}

\item[Platform Difficulty:] the three Post-Architecture cost drivers TIME, STOR and PVOL are here combined together to express the intrinsic complexity of the platform. TIME is a measure of the execution time constraints imposed upon a software system. Since \emph{PowerEnJoy} will offer a service that is expected to be extensively and frequently used by a large customer base, TIME is set to High (4). STOR represents the usage of the available storage offered by the system. As said before, the service provided by the system is supposed to be repeatedly and regularly used. On the other hand, modern technologies can offer a lot of storage capacity in order to prevent saturation, therefore STORE is set to High (4). The PVOL driver expresses the platform volatility. The system will not require frequent major updates, but minor changes, patches and bug fixes are planned to be release constantly in time. Therefore, PVOL is set to Low (2).

The PDIF factor is set to High according to the following table:

\begin{table}[H]
	\begin{adjustwidth}{-.5in}{-.5in}
    \centering
    \begin{tabular}{p{4cm}|p{1cm}|p{1.5cm}|p{1cm}|p{1cm}|p{1cm}}
        \hline
        Sum of TIME, STOR, PVOL Ratings & 8 & 9 & 10-12 & 13-15 & 16-17 \\
        \hline
        \hline
        Rating Levels & Low & Nominal & High & Very High & Extra High \\
        \hline
        Effort Multipliers & 0.87 & 1.00 & 1.29 & 1.81 & 2.61 \\
        \hline
    \end{tabular}
    \caption{PDIF cost driver.}
    \end{adjustwidth}
\end{table}

\item[Personnel Experience:] combines the three Post-Architecture cost drivers APEX, PLEX and LTEX. APEX expresses the level of experience of the project development team. Since the team members have some experience with Java, but never used JEE, the driver is set to Low (2). PLEX describes the productivity influence of the platform experience. As said above, the development team has no experience with technologies supporting JEE, but has some experience with databases and networking. Therefore, PLEX is set to Low (2). LTEX is set to Low (2) too, for the same reasons expressed above, but it must be pointed out that the team members have some basic experience of tools aimed to support requirements modelling and analysis.

The PREX factor is set to Very Low according to the following table:

\begin{table}[H]
	\begin{adjustwidth}{-.5in}{-.5in}
    \centering
    \begin{tabular}{p{4cm}|p{1cm}|p{1cm}|p{1cm}|p{1.5cm}|p{1cm}|p{1cm}|p{1cm}}
        \hline
        Sum of APEX, PLEX, LTEX Ratings & 3,4 & 5,6 & 7,8 & 9 & 10,11 & 12,13 & 14,15 \\
        \hline
        \hline
        Rating Levels & Extra Low & Very Low & Low & Nominal & High & Very High & Extra High \\
        \hline
        Effort Multipliers & 1.59 & 1.33 & 1.22 & 1.00 & 0.87 & 0.74 & 0.62 \\
        \hline
    \end{tabular}
    \caption{PREX cost driver.}
    \end{adjustwidth}
\end{table}

\item[Facilities:] the use of software tools (TOOL) and the multi-site development (SITE) drivers are combined together to express how the infrastructure and tools influences the project. Software tools that will be used in the project are capable, mature, integrated and able to manage the life cycle of the software-to-be, therefore the rating level is set to High (4). The team members usually collaborate residing in the same city. Moreover, internet services (chat, email, video conference...) are frequently used, therefore SITE is set to Very High (5).

The FCIL factor is set to Very High according to the following table:

\begin{table}[H]
	\begin{adjustwidth}{-.5in}{-.5in}
    \centering
    \begin{tabular}{p{4cm}|p{1cm}|p{1cm}|p{1cm}|p{1.5cm}|p{1cm}|p{1cm}|p{1cm}}
        \hline
        Sum of TOOL, SITE Ratings & 2 & 3 & 4,5 & 6 & 7,8 & 9,10 & 11 \\
        \hline
        \hline
        Rating Levels & Extra Low & Very Low & Low & Nominal & High & Very High & Extra High \\
        \hline
        Effort Multipliers & 1.43 & 1.30 & 1.10 & 1.00 & 0.87 & 0.73 & 0.62 \\
        \hline
    \end{tabular}
    \caption{FCIL cost driver.}
    \end{adjustwidth}
\end{table}

\item[Required Development Schedule:] measures the schedule constraints imposed on the project team developing the system-to-be. It is set to Nominal since deadlines are not so flexible and effort must be equally distributed over the given development time.

\begin{table}[H]
	\begin{adjustwidth}{-.5in}{-.5in}
    \centering
    \begin{tabular}{p{3.6cm}|p{1.5cm}|p{1.5cm}|p{1.5cm}|p{1.5cm}|p{1.5cm}|p{1cm}}
    	\hline
        SCED Descriptors: & 75\% of nominal & 85\% of nominal & 100\% of nominal & 130\% of nominal & 160\% of nominal & \\
        \hline
        Rating Levels & Very Low & Low & Nominal & High & Very High & Extra High \\
        \hline
        Effort Multipliers & 1.43 & 1.14 & 1.00 & 1.00 & 1.00 & n/a \\
        \hline
    \end{tabular}
    \caption{SCED cost driver.}
    \end{adjustwidth}
\end{table}

\end{description}

\noindent
The overall results of the analysis of the cost drivers are summarized by the following table:

\begin{table}[H]
    \centering
    \begin{tabular}{l|l|l}
    	\hline
    	Cost Driver & Factor & Value \\
        \hline
        \hline
        Personnel Capability (PERS) & High & 0.83 \\
        \hline
        Reliability and Complexity (RCPX) & High & 1.33 \\
        \hline
        Re-usability (RUSE) & Nominal & 1.00 \\
        \hline
        Platform Difficulty (PDIF) & High & 1.29 \\
        \hline
        Personnel Experience (PREX) & Very Low & 1.33 \\
        \hline
        Facilities (FCIL) & High & 0.73 \\
        \hline
        Required Development Schedule (SCED) & Nominal & 1.00 \\
        \hline
        \textbf{Total}  & $EAF=\prod_i C_i$ & 1.3826 \\
        \hline
    \end{tabular}
\end{table}

\section{Effort Equation}
Equation \ref{effort} allows to estimate the effort in Person-Months (PM):

\begin{equation}
    \textrm{Effort} = A \times EAF \times KSLOC^E
    \label{effort}
\end{equation}

Where the parameters have the following meanings:
\begin{itemize}
    \item $EAF = \prod_i C_i$: derived from the cost drivers analysis.
    \item $E=0.91 + 0.01 \times \prod_{i}SF_i$: derived from the scale drives analysis.
    \item $KSLOC$: Kilo Source Lines of Code estimated via FPs analysis.
    \item $A=2.94$
\end{itemize}

The total effort results in \textbf{31.97} person-months using the value obtained with \ref{avg_e}; the result is of \textbf{47.64} person-months in case the upper-bound estimate \ref{upbound_e} is used.

\section{Schedule Estimation}
The duration of the project is calculated using Equation \ref{duration}.

\begin{equation}
    \textrm{Duration} = 3.67 \times (PM)^{0.28+0.2 \times (E-B)}
    \label{duration}
\end{equation}

\noindent
where B is equal to 0.91 for COCOMO II and PM is the effort calculated by Equation \ref{effort}. The exponent E depends on the scale drivers analysed above.

Using the value of PM obtained with the average estimate of SLOC, the duration results in 10.747 months, corresponding to 2.97 developers.
\noindent
Using the value of PM obtained with the upper-bound estimate of SLOC, the duration results in 12.162 months, corresponding to 3.9 developers.

Since our team is composed only by two people, the estimated development time would be:

\[\frac{31.97\textrm{ pm}}{2 \textrm{ people}}=15.9\textrm{ months}\]

\noindent
or, as upper-bound estimate:

\[\frac{47.64\textrm{ pm}}{2 \textrm{ people}}=23.8\textrm{ months}\]

\noindent
so, the project overall time will be around 16 months, and most probably not greater than 24 months.