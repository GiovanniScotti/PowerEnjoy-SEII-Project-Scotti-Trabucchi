\section{Scale Drivers}
Some of the most important factors contributing to a project's duration and cost are the Scale Drivers. Each of the said drivers describes the project itself and determines the exponent used in the Effort Equation. More precisely, the Scale Drivers reflect the non-linearity of the effort with relation to the number of SLOC. Each scale driver has a range of rating levels from \textit{Very Low} to \textit{Extra High}. The result of the evaluation is summed up in Table \ref{scale_drivers}.

\begin{description}
\item[Precedentedness:] points out the previous experience of the team with the organization and development of large scale projects. Since the team members are new to most of the notions concerning this kind of projects, the Precedentedness is \textit{LOW}.
\item[Development Flexibility:] reflects the level of flexibility in the development process with respect to the given specifications and requirements. Since the project is bounded to a set of prescribed specifications, but a certain degree of flexibility is allowed in the definition of the requirements, the Development Flexibility is \textit{NOMINAL}.
\item[Risk Resolution:] reflects the level of awareness and responsiveness with respect to risks. A rather detailed risk analysis, followed by possible countermeasures, is offered in Section \ref{risk_man}. Therefore, the Risk Resolution driver is set to \textit{HIGH}.
\item[Team Cohesion:] indicates how good the relationship among the team members is and how well the development team works together. Since the team is effective and cohesive with no communication problems, the parameter is set to \textit{VERY HIGH}.
\item[Process Maturity:] by means of a set of levels, it describes how well the behaviours, practices and processes of an organization can reliably produce required outcomes. It is set to \textit{CMM Level 3,} which corresponds to a defined level with respect to the software development process of our organization. At this level the organization has developed its own standard software process and has a greater attention to documentation, standardization and integration. In the COCOMO II model, this corresponds to a level of \textit{HIGH}.
\end{description}

\begin{table}[H]
    \centering
    \begin{tabular}{p{0.45\textwidth}|p{0.35\textwidth}|p{0.1\textwidth}}
        \hline
        \multicolumn{1}{c |}{\textbf{Scale Driver}} & \multicolumn{1}{c |}{\textbf{Factor}} & \multicolumn{1}{c}{\textbf{Value}} \\
        \hline
        \hline
        Precedentedness (PREC) & Low & 4.96 \\
        \hline
        Development Flexibility (FLEX) & Nominal & 3.04 \\
        \hline
        Risk Resolution (RESL) & High & 2.83 \\
        \hline
        Team Cohesion (TEAM) & Very High & 1.10 \\
        \hline
        Process Maturity (PMAT) & High & 3.12 \\
        \hline
        \textbf{Total} & $E=0.91 + 0.01 \times \sum_{i}SF_i$ & 1.0605 \\
        \hline
    \end{tabular}
    \caption{Result of the scale drivers analysis.}
    \label{scale_drivers}
\end{table}

\section{Cost Drivers}
Cost Drivers appear as parameters of the effort equation reflecting characteristics of the developing process and acting as multiplication factors on the effort needed to carry out said project.

Since this effort analysis is carried out at the beginning of the life-cycle of the project, before the writing of the RASD~\cite{rasd} and the DD~\cite{dd}, the point-of-view will be that of the \textit{early design} version of the COCOMO II model. Clear information about the architecture are not still available and the team is exploring and evaluating different architectural alternatives.

In order to conduct the analysis of said cost drivers, guidelines given by the COCOMO Model Definition Manual~\cite{cocomo-manual} have been followed.

Each cost driver of the post-architecture approach has a rating that is considered as a numerical value during the early design analysis (Very Low = 1, Extra High = 6).

%PERS
%RCPX
%RUSE
%PDIF
%PREX
%FCIL
%SCED
\begin{description}
\item[Personnel Capability:] describes the overall capabilities of the team members in terms of problem-solving, actual implementation skills and ratings of personnel turnover. It derives from the conjunction of the Analyst Capability (ACAP) factor, the Programmer Capability (PCAP) factor and the Personnel Continuity (PCON) factor. In details, the ACAP factor expresses the analysis and design ability of the team members, as well as the ability to communicate and cooperate. It is set to Nominal (3) since the team members have not previous experience in finding requirements, but  communication and cooperation is very high. Also PCAP is set to Nominal (3) because the programming abilities of the team is in the average. PCON is instead set at Very High (5) since the turnover is totally absent.

The PERS factor is set to High according to the following table:

\begin{table}[H]
	\begin{adjustwidth}{-.5in}{-.5in}
    \centering
    \begin{tabular}{p{4cm}|p{1cm}|p{1cm}|p{1cm}|p{1.5cm}|p{1cm}|p{1cm}|p{1cm}}
        \hline
        Sum of ACAP, PCAP, PCON Ratings & 3,4 & 5,6 & 7,8 & 9 & 10,11 & 12,13 & 14,15 \\
        \hline
        \hline
        Rating Levels & Extra Low & Very Low & Low & Nominal & High & Very High & Extra High \\
        \hline
        Effort Multipliers & 2.12 & 1.62 & 1.26 & 1.00 & 0.83 & 0.63 & 0.50 \\
        \hline
    \end{tabular}
    \caption{PERS cost driver.}
    \end{adjustwidth}
\end{table}

\item[Product Reliability and Complexity:] this Early Design cost driver depends on the characteristics of the product under development and combines the following four Post-Architecture cost drivers: Required
software reliability (RELY), Database size (DATA), Product complexity (CPLX), and
Documentation match to life-cycle needs (DOCU).
RELY is set to Nominal (3) since a software failure brings moderate financial losses, but there is no risk to human life.

\item[Re-usability:] this cost driver points out the additional effort needed to construct components intended for reuse on current or future projects. A Nominal value is set because re-usability is actually limited to the project itself and no similar projects are planned for the future. 

\begin{table}[H]
	\begin{adjustwidth}{-.5in}{-.5in}
    \centering
    \begin{tabular}{p{3.8cm}|p{1cm}|p{1cm}|p{1.5cm}|p{1.6cm}|p{1.5cm}|p{2cm}}
    	\hline
        RUSE Descriptors: & & none & across project & across program & across product line & across multiple product lines \\
        \hline
        Rating Levels & Very Low & Low & Nominal & High & Very High & Extra High \\
        \hline
        Effort Multipliers & n/a & 0.95 & 1.00 & 1.07 & 1.15 & 1.24 \\
        \hline
    \end{tabular}
    \caption{RUSE cost driver.}
    \end{adjustwidth}
\end{table}

\item[Platform Difficulty:]
\item[Personnel Experience:]
\item[Facilities:]
\item[Required Development Schedule:] measures the schedule constraints imposed on the project team developing the system -to-be. It is set to Nominal since deadlines are not so flexible and effort must be equally distributed over the given development time.

\begin{table}[H]
	\begin{adjustwidth}{-.5in}{-.5in}
    \centering
    \begin{tabular}{p{3.6cm}|p{1.5cm}|p{1.5cm}|p{1.5cm}|p{1.5cm}|p{1.5cm}|p{1cm}}
    	\hline
        SCED Descriptors: & 75\% of nominal & 85\% of nominal & 100\% of nominal & 130\% of nominal & 160\% of nominal & \\
        \hline
        Rating Levels & Very Low & Low & Nominal & High & Very High & Extra High \\
        \hline
        Effort Multipliers & 1.43 & 1.14 & 1.00 & 1.00 & 1.00 & n/a \\
        \hline
    \end{tabular}
    \caption{SCED cost driver.}
    \end{adjustwidth}
\end{table}

\end{description}

\noindent
The overall results of the analysis of the cost drivers are summarized by the following table:

\begin{table}[H]
    \centering
    \begin{tabular}{l|l|l}
    	\hline
    	Cost Driver & Factor & Value \\
        \hline
        \hline
        Personnel Capability (PERS) & & \\
        \hline
        Reliability and Complexity (RCPX) & & \\
        \hline
        Re-usability (RUSE) & & \\
        \hline
        Platform Difficulty (PDIF) & & \\
        \hline
        Personnel Experience (PREX) & & \\
        \hline
        Facilities (FCIL) & & \\
        \hline
        Required Development Schedule (SCED) & & \\
        \hline
        \textbf{Total}  & $EAF=\prod_i C_i$ & \\
        \hline
    \end{tabular}
\end{table}

\section{Effort Equation}
Equation \ref{effort} allows to estimate the effort in Person-Months (PM):

\begin{equation}
    \textrm{Effort} = A \times EAF \times KSLOC^E
    \label{effort}
\end{equation}

Where the parameters have the following meanings:
\begin{itemize}
    \item $EAF = \prod_i C_i$: derived from the cost drivers analysis.
    \item $E=0.91 + 0.01 \times \prod_{i}SF_i$: derived from the scale drives analysis.
    \item $KSLOC$: Kilo Source Lines of Code estimated via FPs analysis.
    \item $A=2.94$
\end{itemize}

\section{Schedule Estimation}