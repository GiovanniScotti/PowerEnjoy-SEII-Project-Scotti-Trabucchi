This chapter is devoted to the Function Point Analysis for the \textit{PowerEnJoy} project, aimed at obtaining a reasonable estimate of the size of the project, which will be later used within the COCOMO II estimation model to compute an average effort factor for the development process.

\section{Internal Logic Files (ILFs)}
Internal Logic Files are defined as follows~\cite{cocomo-manual}:
\begin{quotation}
\textit{"Internal Logic Files count each major logical group of user data or control information in the software system as a logical internal file type. They include each logical file (e.g., each logical group of data) that is generated, used, or maintained by the software system."}
\end{quotation}
\noindent
In practice, they can be identified as a homogeneous set of data used and managed by the application itself.

The identified ILFs for the application are:
\begin{itemize}
\item User
\item Car
\item Payment
\item Reservation
\item Ride
\item AlternativeChargesSituation
\item SafeArea
\item PowerGridStation
\end{itemize}

\section{External Interface Files (EIFs)}
External Interface Files are defined as follows~\cite{cocomo-manual}:
\begin{quotation}
\textit{"Files passed or shared between software systems should be counted as External Interface File types within each system."}
\end{quotation}
In practice, they can be identified as a homogeneous set of data used by the application, but generated and maintained by other applications.

The identified EIFs for the application are:
\begin{itemize}
\item Payment records in the Payment Handlers databases
\item Maintenance intervention records in the Maintenance System database
\item The data streams related to the \textit{Google Maps} service
\end{itemize}

\section{External Inputs (EIs)}
External Inputs are defined as follows~\cite{cocomo-manual}:
\begin{quotation}
\textit{"External Inputs count each unique user data or user control input type that enters the external boundary of the software system being measured."}
\end{quotation}
In practice, they can be identified as elementary operations to elaborate data coming from the external environment.

The identified EIs for the application are:
\begin{itemize}
\item The registration process
\item The login process
\item The update process for user profiles
\item Data streams from the sensors and equipment of cars
\item Data about the availability of cars
\item The car reservation process
\item The car unlocking process
\item The user authentication process for the rides
\end{itemize}

\section{External Outputs (EOs)}
External Outputs are defined as follows~\cite{cocomo-manual}:
\begin{quotation}
\textit{"External Outputs count each unique user data or control output type that leaves the external boundary of the software system being measured."}
\end{quotation}
In practice, they can be identified as elementary operations that generate data for the external environment.

The identified EOs for the application are:
\begin{itemize}
\item E-mail notifications
\item User notifications
\end{itemize}

\section{External Inquiries (EQs)}
External Inquiries are defined as follows~\cite{cocomo-manual}:
\begin{quotation}
\textit{"External Inquiries count each unique input-output combination, where input causes and generates an immediate output."}
\end{quotation}
In practice, they can be identified as elementary operation that involve input and output, without significant elaboration of data from logic files.

The identified EQs for the application are:
\begin{itemize}
\item The visualization of user profile data by the user him/herself
\item The detection of alternative charges situation during rides
\end{itemize}