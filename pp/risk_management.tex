%Risk management
%Problemi di interazione con il payment handler e il maintenance system

The \emph{PowerEnJoy} system project might be threatened by many risks. The analysis of the said risks follows a specific approach that splits them up into three groups: project, technical and business risks.

\section{Project Risks}
Project risks pose a threat to the project plan and make the project schedule slip, increasing the overall costs.
The following risks have been found and analyzed, providing a response strategy. 
\begin{description}
\item[Changes concerning requirements:] during the development of the project the requirements can change unexpectedly. This kind of risk cannot be prevented, but it can be reduced by using a style of programming that takes advantage of reusable and extensible code.
\item[Deadlines are not met:] it can occur that the project requires more time than expected to be carried out. In this case only some functionalities will be offered in a first release while less essential features will be developed later. The above-mentioned first release can provide the target services by the only means of the Mobile Application. The Web Application and the Web Tier may be built afterwards.
\item[Lack of experience in using specific frameworks:] if the team has never used Java Enterprise Edition and has no actual experience in programming with the said framework, the development process will definitely be slowed down. This is that case.
\item[Lack of communication:] team members usually work separately. This can cause misunderstandings and conflicts about the division of the different tasks. To prevent the risk, crystal clear and complete specification and design documents must be provided. Moreover the responsibilities of each group member must be defined in this document.
\item[Team cohesion:]
%Se un membro del team se ne va?
\end{description}

\section{Technical Risks}
Technical risks threaten the quality and the punctuality of the software product preventing a straightforward implementation.

\begin{description}
\item[Unreadable code:] huge and large projects may have a badly structured and unreadable code. Documenting the code can be a reasonable way to mitigate this risk. Furthermore information provided by a good Design Document can come in handy too.
\item[Scalability issues:] if the system does not scale properly with the increasing number of users, a work of major redesign will be carried out. To prevent the issue, a correct estimation of the computing power and the number of involved machines is necessary.
\item[Integration testing failure:]
\item[Downtime:]
\item[Deployment difficulties:]
\item[Data loss and leaks:]
\end{description}

\section{Economical Risks}
Economical risks jeopardize the whole software product threatening its viability.

\begin{description}
\item[Bankruptcy:]
\item[Local regulation and policies:]
\item[Competitors:]
\end{description}